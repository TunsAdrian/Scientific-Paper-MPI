\documentclass[]{article}
\usepackage[utf8]{inputenc}

\title{Review paper 31}
\date{}

\begin{document}	
	
\maketitle

The paper with the title "The Relation between Input Complexity, Size and Time Cost in Quick Sort and Insertion Sort" is in the scope.

Summary: this paper has as a main focus two sorting algorithms: Quick Sort and Insertion Sort, with emphasis on Quick Sort, the paper sustaining the generality and adaptability of this algorithm. It presents the relation between input complexity, size and efficiency of the previously named algorithms, while also making a comparison between Quick Sort, a recursive algorithm and other non-recursive algorithms(Bubble Sort, Selection Sort, Insertion Sort), with emphasis on the Insertion Sort, with the scope of showing the value of Quick Sort as a general purpose sorting algorithm. Both analytic and experimental comparisons were done in this paper. The experiments were done on inputs in a range of 500 to 1.000.000 elements on different input complexity, input complexity being explained in the paper as the constant number of inversions in the permutation corresponding to the certain input, ranging from sorted, to almost sorted, and going as far as completely randomly generated inputs, the later mentioned having the greatest complexity.

The main results seem fairly important. Results are non-trivial and seem original. They also seem intuitively correct, strengthening what was claimed in the first part of the paper. The author uses the notation $\Theta$(theta) without explaining it. The results are represented graphically, and are then properly interpreted in the "Conclusion" section.

The experiment is satisfying, the author offering sufficient details for understanding the concepts presented in the paper, through the analysis of complexity and viability of the algorithms, each step being properly explained. The undergone tests are made for different type of inputs, with emphasis on the worst, average and best case of the algorithms.

The "Related work" section makes a good comparison with literature, in this section also being stated that because the topic of this paper is of great importance and is a vast topic, not all technical details were covered. The "Future work" section covers what has not been yet discussed and what is planned to be addressed in future work.

The author masters the technicalities I have checked thoroughly and also seems to master the field. The paper is according to the standard. Presentation, style and format are of high quality, making the presentation an excellent one. The "References" section contains trustworthy sources that I have verified.

Suggestions: explain the notation $\Theta$(theta) in the paper

Note: the best paper from the three I had.

Decision: strong accept.
\end{document}