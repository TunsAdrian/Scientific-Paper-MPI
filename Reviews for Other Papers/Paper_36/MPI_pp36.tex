\documentclass[]{article}
\usepackage[utf8]{inputenc}

\title{Recenzia lucrării 36}
\date{}

\begin{document}	
	
\maketitle
Lucrarea intitulată "Analiza comparativă a complexității" se află în domeniul de interes pentru acest apel. 

Rezumat: Lucrarea prezintă patru algoritmi de sortare: Bubble Sort, Insertion Sort, Quick Sort și Merge Sort. Algoritmii nu sunt specificați în "Rezumat" sau "Introducere". Sunt prezentate detalii teoretice despre algoritmii respectivi, iar mai apoi sunt afișate rezultatele experimentelor realizate pe secvențe de numere aflate între 1.000 și 100.000 de elemente. Experimentele au fost realizate pe liste de numere sortate în modalități diferite pentru fiecare algoritm, experimentele fiind astfel inconsistente. În ceea ce privește limbajul utilizat, deși e specificat ca fiind doar C++, e prezent și limbajul pseudocod, marcând o altă inconsistență. După rezultatele experimentelor, autorul confirmă elementele teoretice menționate, oferind o explicație vagă. Deși lucrarea propune și comparații teoretice, sunt realizate doar comparații experimentale. Lipsa secțiunii "Concluzie" împiedică exprimarea clară a rezultatului acestei lucrări. Conform problemelor prezentate anterior, lucrarea nu este suficient de completă pentru a putea fi evaluată.

Rezultatele experimentelor par a fi originale și triviale. Lipsa menționării mediului de execuție al experimentelor împiedică exprimarea unei opinii în raport cu corectitudinea rezultatelor.

Autorul nu pare să fie stăpân pe domeniul de cercetare. Deși autorul oferă surse valide în cadrul bibliografiei, nu sunt prezente referințe spre aceste surse, termenii și conceptele utilizate ce aparțin domeniului rămânând nedefinite.

Din punct de vedere al structurii, lipsesc secțiuniile "Lucrări similare", "Modelul și implementarea problemei și a soluției" și secțiunea "Concluzie", iar secțiunea "Descriere formală a problemei și a soluției" e îmbinată cu secțiunea "Studiu de caz". Prezentarea și stilul utilizat lasă de dorit. Titlul nu este un titlu suficient de explicit, nefiind precizată complexitatea a cărui lucru sau fenomen este analizată. Lucrarea nu este realizată în conformitate cu standardul. 

Sunt prezente numeroase secvențe ambigue, fără explicații, precum termenii: "a[i]", "a[j]", "aux", "ok" ce apar menționați în secțiunea 3, fără nici un fel de explicație. De asemenea, sunt prezente numeroase secvențe de cod fără explicații sau comentarii. Sunt prezente numeroase greșeli gramaticale, începănd din secțiunea "Rezumat"(spre exemplu: "teorietic", "domenul", "efciența", "ecinet" etc.).

Sugestii: rescrierea lucrării, plasarea rezumatului pe prima pagina, înainte de cuprins și conceperea și introducerea secțiunilor lipsă, revizuirea titlului, corectarea greșelilor gramaticale, abordarea consistenței în experimente și în limbajul de programare utilizat, explicarea secvențelor de cod, explicarea detaliată a algoritmilor utilizați, preferabil cu explicarea termenilor utilizați(de exemplu: "a[i]"), prezentarea mediului de desfășurare al experimentelor, relizarea unor experimente aprofundate, abordarea comparațiilor teoretice(precum este afirmat în secțiunea "Introducere"), oferirea referințelor spre "Bibliografie".

Decizie: reject(respins).
\end{document}