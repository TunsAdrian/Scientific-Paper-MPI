\documentclass[]{article}
\usepackage[utf8]{inputenc}

\title{Recenzia lucrării 74}
\date{}

\begin{document}	
	
\maketitle

Lucrarea intitulată "Analiza algoritmilor elementari de sortare" se află în domeniul de interes pentru acest apel.

Rezumat: Această lucrare prezintă o parte din algoritmii elementari de sortare: Bubble Sort, Insertion Sort, Selection Sort, Quick Sort și Merge Sort. Lucrarea urmărește atât o analiză teoretică a complexității algoritmilor în raport cu timpul de execuție cât și o analiză experimentală a acestora. Sunt oferite numeroase exemple din viața reală, ce ajută la înțelegerea importanței temei abordate. Experimentele au fost făcute pe numere ce aparțin intervalului [1,50.000], pe seturi de date mici, medii și mari, delimitate în 3 cazuri: șir de elemente generate aleator, șir de elemente sortate descrescător și pe șiruri de elemente aproape sortate(explicat în lucrare drept un șir sortat, dar cu o zecime din elemente interschimbate). Datele de intrare au fost alese astfel pentru a confirma comportamentul algoritmilor în cazurile respective, cuprinzând astfel cazul mediu, cazul nefavorabil și cazul favorabil al algoritmilor studiați în lucrare. Rezultatele arată eficiența algoritmilor Divide et Impera: Merge Sort și Quick Sort, menționând însă dezavantajul memoriei adiționale necesare Merge Sort-ului și cazul nefavorabil al Quick Sort-ului, caz în care se află în aceeași categorie cu algoritmii de bază de sortare. E analizată de asemenea eficiența relativ asemanatoare a algoritmilor Bubble Sort, Insertion Sort și Selection Sort.

Rezultatele par a fi corecte, originale și relevante, fiind de asemenea non-triviale. Rezultatele sunt interpretate în cadrul lucrării prin ilustrații grafice.

Experimentele realizate sunt satisfăcătoare, întrucât autorul oferă suficiente detalii pentru înțelegerea modului de funcționare al algoritmilor prezentați, oferind de asemenea o explicație detaliată pentru înțelegerea complexității algoritmilor și intepretând corespunzător rezultatele obținute în urma experimentelor.

Lucrarea este corectă din punct de vedere tehnic, iar autorul pare a stăpâni câmpul domeniului de interes. Analiza teoretică a complexității este corectă, precum raportează și sursele oferite(surse veridice, precum cartea lui Donald Knuth). 

Prezentarea este bună. Din punct de vedere al structurii, lipsesc secțiuniile: "Lucrări similare". De asemenea, nu se face nici o referire la viitoare lucrări. Fiind prezente atât secvențe de cod scrise în pseudocod, cât și în limbajul python, e prezentă o inconsistență a limbajului de programare utilizat. De asemenea, nu sunt oferite explicații sau comentarii aferente pentru codul prezentat. Bibliografia cuprinde surse veridice, însă nu sunt oferite referințe spre acestea pe parcursul lucrării. 

Sugestii: revizuirea lucrării prin implementarea secțiunii "Lucrări similare" și oferirea referințelor spre bibliografie, scrierea explicațiilor pentru secvențele de cod și încercarea abordării consistenței limbajulu de programare.

Decizie: weak accept(accept slab).

\end{document}